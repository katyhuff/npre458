% use the answers clause to get answers to print; otherwise leave it out.
\documentclass[12pts, answers]{exam}
%\documentclass[12pts]{exam}
\RequirePackage{amssymb, amsfonts, amsmath, latexsym, verbatim, xspace, setspace}

% By default LaTeX uses large margins.  This doesn't work well on exams; problems
% end up in the "middle" of the page, reducing the amount of space for students
% to work on them.
\usepackage[margin=1in]{geometry}
\usepackage{enumerate}
\usepackage{hyperref}

% Here's where you edit the Class, Exam, Date, etc.
\newcommand{\class}{Senior Design}
\newcommand{\term}{Spring 2017}
\newcommand{\assignment}{Design Topic Ideas}
\newcommand{\duedate}{May 2017}
%\newcommand{\timelimit}{50 Minutes}

\newcommand{\nth}{n\ensuremath{^{\text{th}}} }
\newcommand{\ve}[1]{\ensuremath{\mathbf{#1}}}
\newcommand{\Macro}{\ensuremath{\Sigma}}
\newcommand{\vOmega}{\ensuremath{\hat{\Omega}}}

% For an exam, single spacing is most appropriate
\singlespacing
% \onehalfspacing
% \doublespacing

% For an exam, we generally want to turn off paragraph indentation
\parindent 0ex

%\unframedsolutions
\usepackage{bibentry}
\begin{document} 

% These commands set up the running header on the top of the exam pages
\pagestyle{head}
\firstpageheader{}{}{}
\runningheader{\class}{\assignment\ - Page \thepage\ of \numpages}{Due \duedate}
\runningheadrule

\class \hfill \term \\
\assignment \hfill Due \duedate\\
%\begin{flushright}
%\begin{tabular}{p{5in} r l}
%\end{tabular}
%\end{flushright}
\rule[1ex]{\textwidth}{.1pt}

%%%%%%%%%%%%%%%%%%%%%%%%%%%%%%%%%%%%%%%%%%%%%%%%%%%%%%%%%%%%%%%%%%%%%%%%%%%%%%%%%%%%%
%%%%%%%%%%%%%%%%%%%%%%%%%%%%%%%%%%%%%%%%%%%%%%%%%%%%%%%%%%%%%%%%%%%%%%%%%%%%%%%%%%%%%

For \class, you are asked to complete a comprehensive design project. It is 
intended to tie together your nuclear engineering knowledge with your 
engineering design expertise through an independent analysis of design 
considerations for a nuclear system of some kind. 

% ---------------------------------------------

\section*{Topic Examples}
For your guidance, a list of example topics appears here.
Feel free to choose one of these.  However, choosing a creative topic
of your choice is encouraged. This section may be expanded if I have new ideas
as the semester progresses.

\paragraph{Front End Fuel Cycle Support for Advanced Reactors} Many new reactor
design ideas have peculiar fuel needs (TRISO particles, doubly heterogeneous
TRISO pebbles, liquid salt fuels, higher enrichments). What technology gaps
currently exist \& can you solve them? Note that currently, many of these 
involve processes that are prohibitively expensive. Some raise proliferation 
concerns as well.

\paragraph{$^{7}Li$ Isotopic Separation} Isotopic separation (a.k.a.
enrichment) of Lithium 7 is necessary for appropriate neutronic behavior in an
FHR or MSR. How expensive will it be for some of these reactors and what are
the neutronic impacts of $^6Li$ impurities in such a reactor? Can you balance
performance and cost by designing a salt reactor less sensitive to $^6Li$? 

\paragraph{Creative Enrichment Applications} Some enrichment technologies allow 
for enrichment of more than just uranium (for example, plutonium, boron, 
nickel, etc...) what clever uses of enrichment technology might improve our 
world? What isotopes might you separate? How would that work? What enrichment 
methods would you use? What would the separation efficiencies be?  How much 
money would you make? What would you do with the tails? Is there a market?

\paragraph{Cheaper Reprocessing} How can pyroprocessing be made 10x cheaper
than today's PUREX? How can PUREX be made less expensive than it is now?
        
\paragraph{Impact of a Zero Emissions Tax Credit} New York state recently
(summer 2016) implemented a zero emissions tax credit. It was enough to save
the Fitzpatrick nuclear generating station. Were the parameters of that
state-level to be implemented accross the US, either federally or individually
in each state, the risk of owning and building nuclear plants would decrease.
Quantify the change in risk. Predict the impacts to the nuclear industry. Would
it be enough to save at-risk plants? Would we likely see an increase in new
builds? What other impacts might we see?

\paragraph{The Likelihood and Implications of the Duck Curve} The California
Independent System Operators published the ``duck chart.'' This curve,
describing the predicted mid-day overgeneration of grid-bound electricity,
caused by installations of solar, primarily, makes load-following generation
sources or storage methods necessary. A few questions that would make
interesting projects on this topic include:
\begin{itemize}
\item What is the level of alarm appropriate in reaction to this chart and why?
\item Can you suggest a novel strategy that would allow nuclear generation to
be adapted to load follow?
\item What would the impact to nuclear power be if this situation is allowed to
proceed and there is no curtailment of variable generation sources?
\item How could current storage technologies smooth this curve?
\end{itemize}

\paragraph{Liquid vs. Solid Fuelled Molten Salt Reactor Source Term and Release
Pathways} Reactor designs involving molten salts can have either solid fuel or
fluidized fuel. In the fluid fuel case, proponents are often heard to say they
can't melt down because they operate safely at a melted state. Compare the
source term and release pathways of a containment breach in a solid fuelled
salt reactor vs. a fluid fuelled one.

\paragraph{Economics of Uranium Extraction of Seawater} A great deal of
research is being undertaken to lower the costs of uranium extraction from
seawater. Can you replicate or improve the results of a previous calculation of those
costs? To what extent are those results sensitive to uncertain assumptions
(assumptions of a political, technical, or economic nature, it matters not.) 
What design improvements could reduce the costs?

\paragraph{Metrics for Proliferation in the Nuclear Fuel Cycle} Suggest a
metric that captures proliferation concern in a nuclear fuel cycle scenario.

\paragraph{Assessment of Dose To Workers in Reprocessing Schemes} Propose a model for
calculating the dose impact on workers within an arbitrary fuel cycle. Include
dose due to all steps of the fuel cycle, including reprocessing and disposal.
Compare and contrast fuel cycle strategies with and without reprocessing.

\paragraph{Fuel Cycle Transition Scenario} Choose one of the Evaluation Groups
identified by the Fuel Cycle Options Evaluation and Screening. Use a simulator
(e.g. Cyclus, CLASS, or Orion) or your own model to assess the
time-to-transition from our current reactor fleet to 100\% deployment of the
new technology (e.g. SFRs, MSRs, etc.)

\paragraph{Repository Cost Estimation} Suggest a high level waste repository
site. Suggest appropriate waste forms, waste packages, and other disposal
system design features for this geology. Estimate construction rates, loading
rates, transportation costs, and a closing timeline. With these estimates,
defended by analysis, conduct a life cycle cost estimation of your proposed
site.

\nocite{*}
\bibliographystyle{plain}
\bibliography{sr-design}

\end{document}
