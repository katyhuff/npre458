% use the answers clause to get answers to print; otherwise leave it out.
\documentclass[12pts, answers]{exam}
%\documentclass[12pts]{exam}
\RequirePackage{amssymb, amsfonts, amsmath, latexsym, verbatim, xspace, setspace}

% By default LaTeX uses large margins.  This doesn't work well on exams; problems
% end up in the "middle" of the page, reducing the amount of space for students
% to work on them.
\usepackage[margin=1in]{geometry}
\usepackage{enumerate}
\usepackage{hyperref}

% Here's where you edit the Class, Exam, Date, etc.
\newcommand{\class}{NPRE 397}
\newcommand{\term}{Fall 2016}
\newcommand{\assignment}{Design Considerations for Accident Tolerant Fuels}
\newcommand{\duedate}{2016.12.16}
%\newcommand{\timelimit}{50 Minutes}

\newcommand{\nth}{n\ensuremath{^{\text{th}}} }
\newcommand{\ve}[1]{\ensuremath{\mathbf{#1}}}
\newcommand{\Macro}{\ensuremath{\Sigma}}
\newcommand{\vOmega}{\ensuremath{\hat{\Omega}}}

% For an exam, single spacing is most appropriate
\singlespacing
% \onehalfspacing
% \doublespacing

% For an exam, we generally want to turn off paragraph indentation
\parindent 0ex

%\unframedsolutions
\usepackage{bibentry}
\begin{document} 

% These commands set up the running header on the top of the exam pages
\pagestyle{head}
\firstpageheader{}{}{}
\runningheader{\class}{\assignment\ - Page \thepage\ of \numpages}{Due \duedate}
\runningheadrule

\class \hfill \term \\
\assignment \hfill Due \duedate\\
%\begin{flushright}
%\begin{tabular}{p{5in} r l}
%\end{tabular}
%\end{flushright}
\rule[1ex]{\textwidth}{.1pt}

%%%%%%%%%%%%%%%%%%%%%%%%%%%%%%%%%%%%%%%%%%%%%%%%%%%%%%%%%%%%%%%%%%%%%%%%%%%%%%%%%%%%%
%%%%%%%%%%%%%%%%%%%%%%%%%%%%%%%%%%%%%%%%%%%%%%%%%%%%%%%%%%%%%%%%%%%%%%%%%%%%%%%%%%%%%

For \class, your total grade will be earned through a comprehensive design 
project. It is intended to tie together your nuclear engineering knowledge with 
your engineering design expertise through an independent analysis of design
considerations for accident tolerant fuels.  The project will be assessed as
independent design and research work, much like a journal article undergoes
peer review. I will be looking for : 

\begin{itemize}
\item Novelty and Creativity
\item Technical Detail 
\item Analytic Rigor
\item Verifiability
\item Clarity
\item A Conclusion
\end{itemize}

This work will consist of two deliverables, a proposal and a final report. 


% ---------------------------------------------
\begin{questions}
\addpoints
% intro
\question[10] \textbf{Proposal: Due 2016.10.12}

To help determine of reasonable scope, the first step of the project will be a 
proposal outlining the modeling methods and approach for modeling reactivity 
response of fuel that could improve accident tolerance in future reactor 
designs. The focus is to determine the tolerance of fuels during reactor 
accidents and the ability to inhibit an unconstrained temperature increase. 
This will be assessed by calculating the reactivity feedback coefficients of 
fuel with different parameters.  Simulations will be done using Monte Carlo 
methods.  At least one type of fuel type will be characterized, but others may 
be investigated as well.

Once you submit this proposal, I will respond with feedback. Much
like a conference abstract, the proposal should meet the following guidelines:

\begin{itemize}
\item Minimum 500 words.
\item Maximum 1000 words.
\item Two columns.
\item Reasonable margins.
\item 10 pt font or larger.
\item State the question you plan to answer.
\item Summarize the current state of the art in the literature.
\item Motivate the problem, explaining its relevance.
\item Describe the approach and methods you will take to answer the question.
\item Propose an outline of the analysis, software, data, and/or conclusions that will be delivered.
\end{itemize}

\question[90] \textbf{Final Report: Due \duedate}

A final report detailing the findings of the simulations will be presented at
the end of the semester.  The report will outline the parameters varied and the
resulting reactivity changes.  The report will contain the description of the
methods used and proposed follow up work.  Prepare the final document in the
style of a journal article or conference proceedings. It should meet the
following guidelines:

\begin{itemize}
\item Minimum 3000 words.
\item Maximum 10000 words.
\item Two columns.
\item Reasonable margins.
\item 10 pt font or larger.
\item State the question you answered.
\item Comprehensively report and cite the current state of the art in the literature.
\item Motivate the problem, explaining its relevance.
\item Describe the approach, methods, and other elements of your design recommendations.
\item Describe in detail: the analysis, software, data, conclusions produced in this work.
\item Include publication quality graphs and figures.
\item Cite and provide data and code generated for this work sufficient to reproduce the conclusions.
\item Compare this result to previous results in the literature, reinforce the relevance of the work. 
\item Suggest future work.
\end{itemize}

\end{questions}

%\bibliographystyle{plain}
%\bibliography{}

\end{document}
